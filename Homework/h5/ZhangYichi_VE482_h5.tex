\documentclass{article}
\usepackage{bookmark}
\usepackage{color}
\usepackage{listings}
\usepackage{amsmath}
\usepackage{hyperref}
\usepackage{listings}
\usepackage{xcolor}
\usepackage{graphicx}
\usepackage{amsfonts}
\definecolor{dkgreen}{rgb}{0,0.6,0}
\definecolor{gray}{rgb}{0.5,0.5,0.5}
\definecolor{mauve}{rgb}{0.58,0,0.82}

\lstset{
  basicstyle=\footnotesize, 
  numbers=left, 
  numberstyle=\tiny\color{gray}, 
  stepnumber=1,
  numbersep=5pt, 
  backgroundcolor=\color{white},
  showspaces=false,
  showstringspaces=false,
  showtabs=false,
  frame=shadowbox,
  rulecolor=\color{black},
  tabsize=2,
  captionpos=b,
  breaklines=true, 
  breakatwhitespace=false, 
  title=\lstname,
  keywordstyle=\color{blue},
  commentstyle=\color{dkgreen},
  escapeinside={\%*}{*)}, 
  morekeywords={*,...} 
}

\begin{document}
\begin{itemize}
\item {\bf Ex. 1}\\
{\noindent 1. There should not be deadlock happening because as there are three resouces, there should always enough resource for one process to have two.}\\ 

{\noindent 2. n should be at most 5, which can guarantee that at least one process can get the second resource.\\}

{\noindent 3. $x\leq (1000-20\times 35ms-10\times 20ms-5\times 10ms)/4 = 12.5ms$\\}

{\noindent 4. It means that the process will be triggered twice in the whole cycle. It might make sense in the implementing scheduling of processes with different priorities.\\}

{\noindent 5. From the source code it can be done. If a program operates data together with inputing and outputing them, then it might be I/O bound. In the runtime, maybe syscalls such as top which reveals the running status of CPU and I/O devices can help.\\}




\item {\bf EX. 2}\\

{\noindent 1. It should be $$\left[\begin{matrix}7&4&3\\1&2&2\\6&0&0\\0&1&1\\4&3&1\end{matrix}\right]$$}

{\noindent 2. It is in a safe state. Run process as 2,4,5,1,3 can help.\\}

{\noindent 3. Yes, the order above can help.}\\

\item {\bf Ex. 4}\\
This can be done by searching the word "scheduling" in the derictory usr/src. And it can be found in minix/kernel/main.c. In teh code, what I found is that the kernel will check whether certain process is schedulable on startup, and if it is , the kernel will reset its priority and run it. And it can also been seen that root commands and syscalls have higher priority than other processes.

\item {\bf Ex. 5}\\

{\noindent 1. This should be done using count\_lock. Use this to access the counter and change it for reading.}

\begin{lstlisting}[language=C]
void read_lock(){
    down(count_lock);
    if (counter==0) down(db_lock);
    counter++;
    up(count_lock);      
}

void read_lock(){
    down(count_lock);
    if (counter==1) up(db_lock);
    counter--;
    up(count_lock);      
}


\end{lstlisting}
{\noindent 2. It means that the writer cannot find time to write on the db, bercause readers are coming continuously.\\}

{\noindent 3. This can be done by adding down(read\_lock) and up(read\_lock) in both read\_lock() and write\_lock()\\}

{\noindent 4. No as the writer still has higher priority, so the problem should not be regarded as solved.}

\end{itemize}
\end{document}