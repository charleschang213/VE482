\documentclass{article}
\usepackage{bookmark}
\usepackage{color}
\usepackage{amsmath}
\usepackage{hyperref}
\usepackage{listings}
\usepackage{xcolor}
\usepackage{graphicx}
\usepackage{amsfonts}
\begin{document}
\begin{itemize}
\item How to enable built-in debugging in gcc?

When compiling, add the flag -g 

\item What is the meaning of GDB?

GDB gives programmers a tool to check the running progress of their program in a more specified way, which will help them a lot in finding the bugs.

\item What languages are supported by GDB?

It supports Ada, Assembly, C, C++, D, Fortran, Go, Objective-C, OpenCL, Modula-2, Pascal and Rust

\item The commands serves as the following 

\begin{itemize}
    \item backtrace: to show where your program has runned implementation
    \item where : they are additional aliases for backtrace 
    \item finish: it asks the program to finish the current function
    \item delete: it will delete a break point 
    \item info breakpoints: it shows the condition on the line following the affected breakpoint, together with its condition evaluation mode in between parentheses. 
\end{itemize}
\item How to use conditional breakpoints?

Just add a command 'condition breakpoint expression', the breakpoint will only stop when the expression is satisfied.

\item What is -tui option for GDB?

Activate the Text User Interface when starting. The Text User Interface manages several text windows on the terminal, showing source, assembly, registers and GDB command outputs 

\item What is the “reverse step” in GDB and how to enable it. Provide the key steps and commands.

It will run the program backward until control reaches the start of a different source line; then stop it, and return control to GDB. In using, just use 'reverse-step [count]'

\end{itemize}

\end{document}