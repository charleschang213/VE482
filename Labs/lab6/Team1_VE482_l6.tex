\documentclass{article}
\usepackage{bookmark}
\usepackage{color}
\usepackage{amsmath}
\usepackage{hyperref}
\usepackage{listings}
\usepackage{xcolor}
\usepackage{graphicx}
\usepackage{amsfonts}
\begin{document}


\begin{itemize}

\item    {\bf Note: } The lab content is completed by four members of VE482 Project Team 1 , which are 

\begin{itemize}
    \item Zhang Yichi
    \item Zhao Wenhui
    \item Zhou Runqin
    \item Zhu Yuankun
\end{itemize}

\item What is a kernel module, and how does it different from a regular library?

A kernel module is an object file which serves to extend the usage of the kernel, for example, they can add drivers to harwares, install new filesystems, add new syscalls, etc.

Comapred with a a kernel module, a library usually offers some resources that can be used by other programs, instead of offering independent features as kernel module does.

\item How to compile a kernel module?

In the Makefile, you should define the name of object-m, the version of your kernel (KVERSION=), and then use the following line to make 

make -C [module directory for the kernel] M=[working dir] modules


\item Programming part description is in code/README

\item How is information shared between the kernel and user spaces? 

The functions copy\_to\_user() and copy\_from\_user() are used.

\item It is in code/README

\end{itemize}

\end{document}