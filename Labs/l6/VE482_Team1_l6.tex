\documentclass{beamer}
\usepackage{bookmark}
\usepackage{beamerthemeBerkeley}
\usepackage{color}
\usepackage{amsmath}
\usepackage{hyperref}
\hypersetup{colorlinks,linkcolor=blue,citecolor=blue}
\usepackage{listings}
\usepackage{xcolor}
\usepackage{graphicx}
\usepackage{amsfonts}
\usecolortheme{default}
\usetheme{Berkeley}

\definecolor{dkgreen}{rgb}{0,0.6,0}
\definecolor{gray}{rgb}{0.5,0.5,0.5}
\definecolor{mauve}{rgb}{0.58,0,0.82}

\lstset{
  basicstyle=\footnotesize, 
  numbers=left, 
  numberstyle=\tiny\color{gray}, 
  stepnumber=1,
  numbersep=5pt, 
  backgroundcolor=\color{white},
  showspaces=false,
  showstringspaces=false,
  showtabs=false,
  frame=shadowbox,
  rulecolor=\color{black},
  tabsize=2,
  captionpos=b,
  breaklines=true, 
  breakatwhitespace=false, 
  title=\lstname,
  keywordstyle=\color{blue},
  commentstyle=\color{dkgreen},
  escapeinside={\%*}{*)}, 
  morekeywords={*,...} 
}
\title{Lab Presentation \\Question 2.10}
\author{VE482 Team 1}
\institute{UM-SJTU Joint Institute}
\date{\today}

\begin{document}
\begin{frame}
	\titlepage
\end{frame}

\begin{frame}
	\tableofcontents
\end{frame}


\begin{frame}
	\section{Virtual Function}
	\frametitle{Virtual Function}
	Specify a member function as {\color{blue} virtual} for a base class will guarantee the triggering of {\color{blue} dynamic dispatch} in run time if the function is overridden in derived classes
	\begin{block}{Definition}
		{\color{blue} Dynamic dispatch} is the process of selecting which implementation of a polymorphic operation to call at run time.
	\end{block}
	Virtualness of a function can be inherited in a derived class.
\end{frame}

\begin{frame}[fragile]
	\frametitle{Virtual Function Usage}
	\begin{lstlisting}[language=C++]
	class A {
	    ...
	    void func1(){std::cout << "A::func1";}
	    virtual void func2(){std::cout << "A::func2";}
	    ...
	}
	class B: public A {
	    ...
	    void func1(){std::cout << "B::func1";}
	    void func2(){std::cout << "B::func2";}
	    ...
	}
	int main(){
	    std::unique_ptr<A> pt = std::make_unique<B>();
	    pt->func1(); //Outputs "A::func1"
	    pt->func2(); //Outputs "B::func2"
	}
	\end{lstlisting}

	
\end{frame}

\begin{frame}
	\section{Abstract Class}
	\frametitle{Abstract Class}
	\begin{block}{Definition}
		An {\color{blue} abstract class} is a generic class used as a basis for creating different objects that conform to its protocol, or the set of operations it supports.
	\end{block}
	An abstract class usually contains 
	\begin{itemize}
		\item Some member data and methods which are commonly shared by its derived objects
		\item Some methods declared but are not implemented(abstract function,usually declared as {\color{blue} virtual type fname(args)=0}) 
	\end{itemize}
	{\color{blue} Note: }An abstract class cannot be instantiated but a pointer to it can own an instance of its derived class, through which polymorphism is achieved.
\end{frame}

\begin{frame}
	\section{Friend Keyword}
	\frametitle{Friend Keyword}
	The keyword "friend" is used to grant access of private members of a class. Usually it can be used in two diferent aspects.
	\begin{itemize}
		\item Declare a non-member function in a class as a friend.(Often in operator overloading)
		\item Declare another class as a friend.
	\end{itemize}
	\begin{block}{Risk of Using Friend}
		Using a friend keyword means that something outside a specific class can access the "internal" world of a class, which might violates the principle of {\color{blue} encapsulation}.
	\end{block}
	
\end{frame}

\begin{frame}
	\section{Override Specifier}
	\frametitle{Override Specifier}
	Features
	\begin{itemize}
		\item Override specifier should be added right after the function declaration in a derived class ({\color{blue}type funcname() override;})
		\item This keyword will guarantee that this function will be virtual and will override a virtual function in base class
		\item If any of the requirements is not satisfied, the program will not compile
	\end{itemize}
\end{frame}

\begin{frame}
	\section{Final Specifier}
	\frametitle{Final Specifier}
	Features
	\begin{itemize}
		\item Override specifier should be added right after the function declaration in a derived class ({\color{blue}type funcname() final;})
		\item This keyword will guarantee that this function will override a virtual function in base class {\color{blue} but cannot be overridden again}
		\item If any of the requirements is not satisfied, the program will not compile
	\end{itemize}
\end{frame}

\begin{frame}
	
	\begin{block}{Purpose of the two specifiers}
		1. Avoid potential mistakes of developers who might forget to implement an override or have something wrong in overriding.\\

		\vskip 2em

		2. Make the implementation more explicit to possible readers.
	\end{block}
\end{frame}


\begin{frame}
	\section{References}
	\frametitle{References}
	\begin{itemize}
		\item \url{web.mit.edu/merolish/ticpp/Chapter15.html}
		\item \url{en.wikipedia.org/wiki/Dynamic_dispatch}
		\item \url{www.techopedia.com/definition/17408/abstract-class}
		\item \url{en.cppreference.com/w/cpp/keyword/friend}
		\item \url{www.cprogramming.com/tutorial/friends.html}
		\item \url{en.cppreference.com/w/cpp/language/override}
		\item \url{en.cppreference.com/w/cpp/language/final}
	\end{itemize}
\end{frame}

\end{document}